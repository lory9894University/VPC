\documentclass{article}
\usepackage[utf8]{inputenc}

\title{Esercizio Definizioni}
\author{Ruben Castelluccio}
\date{Marzo 2021}

\usepackage{natbib}
\usepackage{graphicx}
\usepackage{amsmath,amssymb}
\newcommand{\N}{\mathbb N}
\begin{document}

\maketitle

\section{Notazioni Numeriche}
\textbf{Numeri Naturali}
\\
L’insieme dei numeri naturali è composto da tutti i numeri interi maggiori o uguali a 0. L’insieme è indicato da:

\begin{equation}
  \overrightarrow{\dim} = \{0,1,2,3,..\}^1,
\end{equation}
\textbf{Numeri Interi}
\\
L’insieme dei numeri interi è composto da tutti i numeri che possono essere scritti senza una componente frazionaria. Comprendono i numeri naturali e gli interi negativi. L’insieme è indicato da:
\begin{equation}
  \mathbb{Z} = \mathbb{N} \cup \{ -n | n \in \mathbb{N} \ \{0\}\},
\end{equation}
\textbf{Numeri Razionali}
\\
L’insieme dei numeri reali è composto da tutti i numeri rappresentanti una quantità sulla linea dei Reali. Comprendono i numeri Interi, i numeri Razionali e i numeri Irrazionali. L’insieme è indicato da:
\begin{equation}
\mathbb{Q} = \{ {m \over n} | m,n \in \mathbb{Z}\}
\end{equation}
\textbf{Numeri Irrazionali}
\begin{equation}
\mathbb{I} = \{ x | x \ne {m \over n}  \forall m,n \in \mathbb{Z}\}
\end{equation}
\textbf{Numeri Reali}
\begin{equation}
  \mathbb{R} = \mathbb{Q} \cup \mathbb{I},
\end{equation}
\section{Notazioni Insiemistiche}
\textbf{Intersezione}
\\
L’intersezione di due insiemi A e B è l’insieme che contiene gli elementi comuni ad A e a B.
\begin{equation}
  A \cap B \iff \{x | x \in A \wedge x \in B\},
\end{equation}
\textbf{Unione}
\\
L’unione di due insiemi A e B è l’insieme che contiene tutti gli elementi di A e B.
\begin{equation}
  A \cup B \iff \{x | x \in A \ \lor x \in B\},
\end{equation}
\textbf{Differenza}
\\
La differenza tra l’insieme A e l’insieme B è l’insieme che contiene tutti gli elementi di A che non sono presenti in B.
\begin{equation}
  A \setminus B \iff \{x | x  \in A \wedge x \not \in B\},
\end{equation}
\textbf{Insieme Potenza}
\\
Il Power Set dell’insieme S è l’insieme che contiene tutti i sottoinsiemi di S, compresi l’insieme vuoto e S stesso. L’insieme è indicato da: 
\begin{equation}
  P(X) \iff \{Y | Y \subseteq B\},
\end{equation}
\textbf{Complemento}
\\
L’insieme complemento di A è l’insieme che contiene gli elementi non presenti in A in un insieme universo U che contiene tutti gli elementi considerati.
\begin{equation}
  X^C \iff \{x | x \not \in X\},
\end{equation}
\\
\textbf{Elemento contenuto in un insieme}
\\
L'appartenenza di un elemento ad un insieme è un concetto che esprime datp un elemento a esso sia compreso tra gli elementi di un insieme $X: a \in X.$ 

\textbf{Sottoinsieme}
\begin{center}

\begin{equation}
  A \subseteq B \iff \forall a \in A,a \in B,
\end{equation}
\end{center}

\textbf{Sottoinsieme Stretto}
\begin{equation}
  A \subset B \iff \forall a \in A,a \in B \wedge \exists b \in B | b \not \in A
\end{equation}
\textbf{Prodotto Cartesiano}
\\
Il prodotto cartesiano tra due o più insiemi è un insieme contenente tutte le coppie ordinate (a, b) in cui a appartiene ad A e b appartiene a B.
\begin{equation}
  A \times B \iff \{ \big(a,b \big) |a \in A \wedge b \in B \}
\end{equation}
\section{Definizioni Algebriche}
\textbf{Relazione Binaria}
Elenco di coppie ordinate di elementi appartenenti al prodotto cartesiano.
\begin{equation}
 xRy \iff \big( x,y \big) \in R
 \end{equation}
\textbf{Proprietà Riflessiva}
Dato un insieme X è detta relazione riflessiva R se ogni elemento appartenente a X è in relazione con se stesso.
\begin{equation}
\forall x \in X, xRX
\end{equation}
\textbf{Proprietà Simmetrica}
\begin{equation}
\forall a,b \in X, aRb \Rightarrow bRa
\end{equation}
\textbf{Proprietà Transitiva}
\begin{equation}
\forall a,b,c \in X,aRb \wedge bRc \Rightarrow aRc
\end{equation}
\textbf{Relazione di Equivalenza}
\\
Una relazione di equivalenza è una relazione riflessiva, simmetrica e transitiva.
\\
\textbf{Chiusura transitiva di una Relazione}
\begin{equation}
R^+ = R \cup \{\big( a,b \big) | \exists c \big( \big( a,c \big) \in R^+ \wedge \big( c,b \big) \in R^+ \big) \}
\end{equation}
\textbf{Funzione}
\\
Una funzione è una relazione tra un set di input e un set di output, con la proprietà che ogni input è relativo ad un solo output.
\begin{equation}
f: X \rightarrow Y \Rightarrow \{\big( x,y \big) \in X \times Y | \forall w,z \big( x,y \big) \in f \wedge \big( x,z \big) \in f \Rightarrow w = z \}
\end{equation}
\textbf{Funzione n-aria}
\begin{equation}
f: X_1 \times ... \times X_n \rightarrow Y \Rightarrow \{ \big(x_1,...,x_n,y \big) \in X_1 \times ...\times X_n \times Y |
\forall w,z \big(x_1,...,x_n,w \big) \in f \wedge \big(x_1,...,x_n,z \big) \in f \rightarrow w = z \big) \}
\end{equation}
\textbf{Funzione Inietttiva}
\\
Una funzione è detta iniettiva se mappa sempre un elemento del suo dominio ad uno ed un solo elemento del suo codominio.
\begin{equation}
  \forall x,y \in X \big( x \not = y \Rightarrow f\big( x \big) \not = f \big( y \big) \big)  
\end{equation}
\textbf{Funzione Suriettiva}
\begin{equation}
    \forall y \in Y, \exists x \in X | f \big( x \big) = y
\end{equation}
\textbf{Funzione Biettiva}
\\
Una funzione Biettiva è sia Iniettiva sia Suriettiva
\section{Definizioni Informatiche}
\subsection{Linguaggi Formali}
\textbf{Alfabeto}\\
Un alfabeto è un insieme i cui membri sono simboli, che possono includere lettere, caratteri e numeri.
Se L è un linguaggio formale, ossia un set finito o infinito di stringhe di finita lunghezza, allora l’alfabeto di L, indicato con Σ, è l’insieme di tutti i simboli che possono comparire in una qualunque stringa di L.
\\
\textbf{Lettera}\\
Un elemento di un alfabeto.
\\
\textbf{Stringa}\\
Una stringa è un elemento del tipo $s_1s_2$...$s_n$, con $s_i$ $\in$ $\sum$, $\forall$ i=1,...,n.
\\
\textbf{Stringa Vuota}\\
Stringa vuota si definisce come elemento neutro rispetto all'operazione $\cdot$
\\
\textbf{Concatenazione}\\
Dato un un insieme di stringhe $S_\sum$, si definisce concatenazione l'operazione $\cdot$ : $S^2_\sum$
$\rightarrow S_\sum$ con $s_1$ $\cdot$ $=$ $s_1$ $s_2$.
\\
\textbf{Ripetizione}
\\Data una stringa s, si defnisice ripetizione di tale stringa s$^n$ = $\cdot$ $^n_i$=1 s
\\
\textbf{Prefisso}
\\
Data s $=$ $s_1$ $\cdot$ $s_2$, con $\cdot$ definita prima, si dice che $s_1$ è un prefisso di s.
\\
\textbf{Suffisso}
\\
Data s $=$ $s_1$ $\cdot$ $s_2$, con $\cdot$ definita prima, si dice che $s_2$ è un suffisso di s.

\subsection{Grafi}
\textbf{Definizione}
\\
Un grafo è una struttura matematica che rappresenta un insieme di oggetti (vertici) in cui alcune coppie di essi sono connesse da archi.
Un grafo è quindi una coppia ordinata G = (V,E) che comprende un insieme V di vertici e un insieme E di archi, che sono associati a V come subset di due elementi di V.
\textbf{Grafo Diretto}
\\ Un grafo diretto G $=$ $\big($ V,E $\big)$ è una coppia odinata, costituita da un insieme di veritici V, detti anche nodi o punti, e un insieme di coppie ordinate di vertiici E chiamate archi diretti.
\\
\textbf{Grafo Indiretto}
\\ Un grafo diretto G $=$ $\big($ V,E $\big)$ è una coppia odinata, costituita da un insieme di veritici V, detti anche nodi o punti, e un insieme di coppie non ordinate di vertiici E chiamate archi indiretti.
\\
\textbf{Grafo Bipartito}
\\
Un grafo bipartito G $=$ $\big($ U,V,E $\big)$ è un grafo in cui U e V rappresentano due insiemi disgiunti di veritici tali per cui $\forall$u,u' $\in$ U $\forall$ v,v' $\in$ V $\not$ $\exists$ $\big($ u,u' $\big)$ $\in$ E $\wedge$ $\not$ $\exists$ $\big($ v,v' $\big)$ $\in$ E.
\\
\textbf{Nodo Sorgente}
\\
Dato un arco $\big($ x,y $\big)$ in un grafo G, x è il nodo sorgente.
\\
\textbf{Nodo Destinazione}
\\
Dato un arco $\big($ x,y $\big)$ in un grafo G, y è il nodo destinazione.
\\
\textbf{In-degree di un nodo}
\\
Dato un grafo G, si definisce in-degree di un nodo x la cardinalità dell'insieme X definito come X $=$ $\{$ $\big($ y,x $\big)$ $|$ $\big($ y,x $\big)$ $\in$ E,y $\in$ V $\}$
\\
\textbf{Out-degree di un nodo}
\\
Dato un grafo G, si definisce out-degree di un nodo x la cardinalità dell'insieme X definito come X $=$ $\{$ $\big($ x,y $\big)$ $|$ $\big($ x,y $\big)$ $\in$ E,y $\in$ V $\}$
\\
\textbf{Funzione di Etichettatura degli Archi}
\\
Dati un insieme di etichette A ed un grafo G, la funzione di etichiettatura degli archi è definita come f : E $\rightarrow$ $\forall$ e $\in$ E.
\\
\textbf{Funzione di Etichettatura dei Nodi}
\\
Dati un insieme di etichette A ed un grafo G, la funzione di etichiettatura dei nodi è definita come f : V $\rightarrow$ A $\forall$ v $\in$ V.
\\
\textbf{Cammino}
\\
Un cammino tra due nodi x$_0$ e x$_n$ è una sequenza ordinata finita di archi distinti del tipo $\bigl\langle$ $\big($x$_0$, x$_1$ $\big)$, ..., $\big($ x$_i_-_1$, x$_i$ $\big)$, $\big($ x$_i$, x$_i_+_1$ $\big)$,..., $\big($ x$_-_1$, x$_n$ $\big)$ $\bigr\rangle$ con n $\geq$ 1.
\\
\textbf{Ciclo}
\\
Un ciclo è un cammino, inteso come sequenza di vertici ed archi, in cui nessun vertice si ripete eccetto il primo (che è anche l’ultimo). Un ciclo di lunghezza n è detto n − ciclo.
Si nota che i sottoinsiemi di V,E che formano il ciclo all’interno del grafo G = (V,E) a loro volta identificano un grafo ciclico.
\\
\textbf{Grafo Fortemente Connesso}
\\
Un grafo si dice fortemente connesso se $\forall$ u,v $\in$ V $\exists$ $\big($ u,v $\big)$ $\in$ E.
\\
\textbf{Componente Fortemente Connessa Terminale}
\\
Una componente fortemente connessa terminale (bSCC) di un grafo è un sottografo fortemente connessa senza archi uscenti.
\\
\textbf{BSS - Bottom Strongly Connect Component}
Una BSCC è una componente fortemente connessa da cui nessun vertice al di fuori della BSCC è raggiungibile.
\\
\textbf{Albero}
\\
Un albero è un grafo indiretto in cui ogni coppia di vertici è connessa da solo un arco. Per questo, ogni grafo indiretto, connesso e aciclico è un albero.
\\
\subsection{Matrici}
\textbf{Matrice n-dimensionale}
\\
Dato un campo C, una matrice di dimensione n a valori di C è un vettore quadrato del tipo:
$$A = \left (
\begin{array}{ccc}
a_1_1 & ... & a_1_n \\
... & ... & ... \\
a_n_1 & ... & a_n_n \\
\end{array}
\right )$$
dove a$_i_j$ $\in$ C.
\\
\textbf{Somma Matriciale}
\\
Date 2 Matrici A e B di M righe e N colonne, si definisce matrice somma C $=$ A $+$ B la matrice i cui elementi c$_i_j$ $=$ \big( a $+$ b \big) $_i_j$ sono dati da: c$_i_j$ $=$ \big(a $+$ b\big) $_i_j$ $=$ a$_i_j$ $+$ b$_i_j$, con a$_i_j$ elemento di A di riga i colonna j, e b$_i_j$ analogo di B. L'operazione $+$ : M$^2_$\big$_m_n_$\big \rightarrow M$_$\big$_m_n_$\big che, data una coppia di matrici siffatta, produce una matrice somma, è detta somma matriciale.
\\
\textbf{Prodotto Matriciale}
\\
Data una matrice A di M righe e N colonne ed una B di N righe e P colonne, si definisce prodotto C $=$ A $\cdot$ B la matrice, di M righe e P colonne, i cui elementi sono dati da: c$_i_j =$ \sum$^N_k_=_1$ a$_i_kb_i_k$, con i $=$ i,..., m e j $=$ 1,...,p. L'operazione \cdot : M$_$\big($_m_n_$\big) \times M$_$\big($_n_p_$\big) \rightarrow M$_\big(_m_p_\big)$ che, data una coppia di matrici siffatta produce una matrice prodotto , è detta prodotto matrciale.
\\
\textbf{Prodotto Vettore per Matrice}
\\
Dato che possiamo definire un vettore come una matrice in cui una delle dimen- sioni è 1, possiamo anche definire il prodotto tra una matrice A e un vettore x. Il prodotto tra matrice e vettore è definito solo nel caso in cui col(A) = rows(x), per cui dobbiamo utilizzare il vettore come se fosse una matrice colonna, ossia una matrice in cui il numero di colonne è 1.
Per calcolare il prodotto tra A e x possiamo considerare lo stesso procedimento definito per il prodotto tra una matrice A e una generica matrice B. Otteniamo quindi di dover calcolare il prodotto scalare di x con ognuna delle righe di A. Il risultato di una moltiplicazione tra una matrice A di dimensioni n × m e un vettore x di dimensioni m × 1 è un vettore colonna di dimesioni n × 1.
\\

\end{document}
