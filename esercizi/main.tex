% \documentclass{article}
% \usepackage[utf8]{inputenc}

% \title{Esercizio 1 definizioni}
% \author{susi }
% \date{March 2022}

% \begin{document}

% \maketitle

% \section{Introduction}

% \end{document}
\documentclass{article}
%\usepackage{times}


%\usepackage{amsthm}
\usepackage{algorithm} % for pseudo-codes
%\usepackage{enumerate}
\usepackage{algpseudocode}
%\usepackage{DotArrow}
%\usepackage{subfig}
%\usepackage{amsmath, environ}
%\usepackage{amssymb}
%\usepackage{amsthm}
%%\usepackage{bm}
%%\usepackage{mathtools}
\usepackage{graphicx} 
%\usepackage{wrapfig}
%\usepackage{minibox}
%\usepackage{multirow}
%\usepackage{pifont} % \ding{}
%\usepackage[percent]{overpic}
%\usepackage{scrextend} % labeling environment
%
%\newtheorem{theorem}{Theorem}
%\newtheorem{definition}{Definition}
%\newtheorem{example}{Example}
%
%\renewcommand{\textfraction}{0.05}
%\renewcommand{\floatpagefraction}{0.75}

\newcommand{\centerimage}[5][-7pt]{ % [vskip] | graphics-opt | imgname | label | caption
\begin{figure}[!htb]%
\centering%
\vspace{#1}%
\includegraphics[#2]{#3}%
\caption{#5}\label{#4}\vspace{2mm}%
%\vspace{-3pt}\caption{#5}\label{#4}\vspace{-2pt}%
\end{figure}}

%\includegraphics[trim=1cm 2cm 3cm 4cm, clip=true]{example.pdf}



\begin{document}



\title{Esercizio definizioni}


\author{autore, email}

\maketitle


%========================================================================= 
%%%%%%%                       INTRODUCTION                         %%%%%%%
%========================================================================= 

\section{Inizio}  \label{sec:inizio}

%=========================================================================================================
\subsection{Primo esercizio}  \label{ssec:primo}
Primo esercizio

\begin{equation} \label{eq:xii}
	xi_i = (I - T_i)^{-1} \cdot
			\sum^m_{j=i+1} (F_i \cdot xi_j) \qquad i \leq k
\end{equation}%


A CSLTA  property over the set $AP$ of atomic proposition is defined as
\begin{equation} 
  %	\text{\emph{(\CSLTA\ state formula)}\qquad}
	\Phi ::= p ~|~ \neg\Phi ~|~ \Phi \wedge \Phi ~|~ 
			 %\mathcal{S}_{\bowtie \lambda}(\Phi) ~|~ 
			 \mathcal{S}_{\bowtie\lambda}(\Phi)  ~|~
			 \mathcal{P}_{\bowtie\lambda}(\mathcal{A})  \qquad~
\end{equation}



La Figura~\ref{img:reteA}  

%\centerimage{trim=1cm 10cm 3cm 4cm, clip=true}
	\centerimage{width=\columnwidth}
{ReteLettoriScrittori.pdf}{img:reteA}
	{Una rete}



%===================== TEST 2 FMS 
\begin{table}
	\centering
	%\includegraphics[width=\columnwidth]{imgs/ztest2.pdf}
	\caption{A table}
	\label{tab:reteA}
\end{table}
%===================== 

La Tabella~\ref{tab:reteA} 



\begin{algorithm}
  \begin{algorithmic}
    \Function{ModelCheck}{$\{Z_j\}$ components of ${Z}$, $s_0$ : initial state}
      \State $\pi^{(0)} : S\times Z \rightarrow {R}$
         \Comment{sparse vector of state probabilities}
      \For{each  $Z_j$,  taken in forward topological order} %, with $0 < j \leq K$}
         \State $H_j =$ all the tuples $(s,z)$ with 
                  $\pi^{(j-1)}[(s,z)] \neq 0 \,\wedge\, z \in Z_j$ 
      \EndFor
      \State\Return{$\pi^{(K)}[\top]$}
    \EndFunction
  \end{algorithmic}
  \caption{Pseudocode of ......}\label{algo:fwd:otf}
\end{algorithm}














\end{document}
