\documentclass[a4paper]{article}
\usepackage{listings}
\usepackage{qtree}
\usepackage{xcolor}
\usepackage{forest}
\usepackage{multicol}
\setlength{\columnsep}{3cm}
\usepackage{parskip}
\usepackage{changepage}
\usepackage[T1]{fontenc}
\usepackage{amsmath}
\usepackage{hyperref}
\usepackage{listings}
\usepackage{amsthm}
\usepackage{amssymb}
\usepackage{float}
\usepackage[utf8]{inputenc}
\usepackage{graphicx}
\usepackage[italian]{babel}
\usepackage{thmtools}
\begin{document}

\author{Lorenzo Dentis, lorenzo.dentis@edu.unito.it}
\title{Esercizi con Uppaal}
\maketitle
%http://ppedreiras.av.it.pt/resources/empse0809/slides/TheUppaalModelChecker-Julian.pdf
\section{Modello A}
Stop and wait e canale perfetto.\\ 
Si assume che il canale sia perfetto, e quindi nè il messaggio, nè l’ack possono essere persi. 
Il tempo di trasmissione sul link è variabile all’interno di un intervallo limitato (costanti minTransmissionTime e maxTransmissionTime ), con una differenza $\leq \fract{1}{10}$ tempo di trasmissione. 

Si definiscano e provino le proprietà di corretto funzionamento del protocollo, inparticolare si provi qual è il tempo minimo e massimo che intercorre dalla spedizione di un messaggio alla ricezione del suo ack.
\subsection{mittente}

\subsection{destinatario}
\subsection{canale}
\section{Modello B}
\section{Modello c}
%controllare anche di non poter spedire il frame 1 senza aver spedito il frame 0
\end{document}
