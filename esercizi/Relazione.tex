
\documentclass{article}   
\usepackage[utf8]{inputenc}

\usepackage{color}
\usepackage{graphicx} % in the STTT example
\usepackage{amsmath}
\usepackage{amssymb}
\usepackage{mathtools}
\usepackage{algorithm}
\usepackage{multirow}
\usepackage{color}
%\usepackage{amsmath,amssymb,dsfont}
\usepackage{multirow}
\usepackage{algpseudocode}
\usepackage{algorithmicx}
\usepackage{dblfloatfix}

\renewcommand{\textfraction}{0.05}
\renewcommand{\floatpagefraction}{0.90}


\newcommand{\red}[1]{\textcolor{red}{#1}}
\newcommand{\blue}[1]{\textcolor{blue}{#1}}

\newcommand{\IGNORE}[1]{}

\newcommand{\ttit}[1]{{\text{\it #1}}}
\newcommand{\imgchar}[1]{{\bf\small({#1})}}
%\newcommand{\rewrite}{Rewrite}
\newcommand{\Sat}{\text{\it Sat}}
\newcommand{\SatELTL}{\text{Sat{$\exists$}LTL}}
\newcommand{\SatLTL}{\text{SatLTL}}
\newcommand{\SatCTL}{\text{SatCTL}}
\newcommand{\SatCTLstar}{\text{SatCTL*}}
\newcommand{\satCTL}{\Call{satCTL}}
\newcommand{\satLTL}{\Call{satLTL}}
\newcommand{\satELTL}{\Call{sat{$\exists$}LTL}}
\newcommand{\rewrite}{\Call{rewrite}}
\newcommand{\tuple}[1]{{\langle{#1}\rangle}}
\newcommand{\CTLstarMEDD}{{\small RGMEDD*}}
\newcommand{\RGMEDD}{{\small RGMEDD3}}
\newcommand{\ltsmin}{LTSmin}
\newcommand{\greatspn}{GreatSPN}
\newcommand{\meddly}{Meddly}
\newcommand{\spot}{Spot}
\newcommand{\Paths}{\mathbf{\cal P}}
\newcommand{\place}[1]{#1}
\newcommand{\trans}[1]{#1}
\newcommand{\Lang}{\mathcal L}
\newcommand{\AP}{AP}


\newcommand{\PsiRule}{\boldsymbol\Psi}
\newcommand{\phiRule}{\boldsymbol\phi}
\newcommand{\varphiRule}{\boldsymbol\varphi}
\newcommand{\doubleAmp}{\ensuremath{\mathop{\scalebox{0.80}{\&\!\&}}}}
\newcommand{\doublePipe}{\ensuremath{\mathop{\scalebox{0.80}{$||$}}}}

\begin{document}
\title{Esercizio Reti di Petri Produttori e consumatori}
\author{il vostro nome e matricola}



   

\date{\today}

\maketitle

\section{Primo setting: 1 produttore, 1 consumatore, 1 buffer a N posizioni}\label{SEC:primo}
L'esercizio chiede di.....

La rete in Figura~\ref{FIG:primo} mostra .....

\begin{figure*}[h!]
\centering
%\includegraphics[width=\textwidth]{ReteIniziale.pdf}
\caption{Una prima rete incompleta} \label{FIG:primo}
\end{figure*}


\section{Secondo  setting: 1 produttore, 2 consumatori,  1 buffer a N posizioni}\label{SEC:secondo}
L'esercizio chiede di.....
\subsection{Secondo  setting: scalatura di marcatura}\label{SEC:secondo-marking}
La soluzione con la scalatura di marcatura richiede di modificare la marcatura del posto \place{$P_0$} ......
\subsection{Secondo  setting: replicazione di sottoreti}\label{SEC:secondo-replica}


\section{Terzo  setting: $P$ produttori, $C$ consumatori,   1 buffer a N posizioni}\label{SEC:terzo}
\subsection{Terzo  setting: scalatura di marcatura}\label{SEC:terzo-marking}
\subsection{Terzo  setting: replicazione di sottoreti}\label{SEC:terzo-replica}


\end{document}
